%
% File acl2013.tex
%
% Contact  navigli@di.uniroma1.it
%%
%% Based on the style files for ACL-2012, which were, in turn,
%% based on the style files for ACL-2011, which were, in turn, 
%% based on the style files for ACL-2010, which were, in turn, 
%% based on the style files for ACL-IJCNLP-2009, which were, in turn,
%% based on the style files for EACL-2009 and IJCNLP-2008...

%% Based on the style files for EACL 2006 by 
%%e.agirre@ehu.es or Sergi.Balari@uab.es
%% and that of ACL 08 by Joakim Nivre and Noah Smith

\documentclass[11pt]{article}
\usepackage{acl2013}
\usepackage{times}
\usepackage{url}
\usepackage{latexsym}
%\setlength\titlebox{6.5cm}    % You can expand the title box if you
% really have to

\title{Instructions for ACL-2013 Proceedings}

\author{Cedric Takongmo \\
  {\tt cedric.takongmo@mni.thm.de}}

\date{}

\begin{document}
\maketitle
\begin{abstract}
Get a concise introduction to Jasmine, the popular testing framework for JavaScript. This document shows you how to write unit tests with Jasmine that automatically check for bugs in your application using the Test-Driven development pattern and the Karma test runner. After you get an overview of test-driven development, learn how to write specifications for individual components, and then use those specs to test the code you write. Write useful specs by determining what you need to test and what you don’t. Test the behavior of new and existing code against the specs you create. Apply Jasmine matchers and discover how to build your own. Organize code suites into groups and subgroups as your code becomes more complex. Use a Jasmine spy in place of a function or an object and learn why it’s valuable.
\end{abstract}

\section{Credits}

This document has been wrote on the basis of some external Internet links and a presentation of Christopher Bartling about JavaScript tests with Jasmine and Karma. The statistics to illustrate the importance of JavaScript today come from studies of diverse companies like TIOBE software. A detailed presentation of the Jasmine library was inspired by the online documentation on the official Jasmine website. 

\section{Motivation for Javascript TDD}

Today, Javascript is growing in popularity and will continue to grow. As show in the Abbildung 1, from 1996 to 2016, JavaScript is from the 23th to the 7th position of the most used programming languages. JavaScript is also used all over the place, and is used even with every backend language. This programming language is also the most commonly used in frontend development to dynamically give more user experience to web applications. JS-libraries are used for the development of web and mobile applications with JQuery Mobile/PhoneGap. Famous libraries like Angular.js, jQuery, Node.js, React and various others tools inherit JavaScript. Some hardware engineer are using JavaScript to do embedded programming so that JavaScript can be used for the direct communication with the outside world by using MicroControllers and other electronics related stuff, although this is definitely less common at the moment. This is why JavaScript is such a good language to learn. (https://www.quora.com/When-do-developers-use-JavaScript-and-why).

A software product has to be considered like every other commercial product and it quality must be tested. This is also valid for javaScript Software. In the software engineering, we are talking about Code Quality. Code quality is a loose approximation of how long-term useful and long-term maintainable the code is. If the Code is thrown away tomorrow is mean that this Code has a Low quality. High quality Code is being carried over from product to product, developed further, maybe even open sourced after establishing is value. The software quality is defined with a clear and understandable design and implementation and also well defined interfaces. Good Code is ease to build and use and have to be extensible. The minimum extra dependencies and the good documentation are very important criteria for Software Quality. One off he most important criteria are the tests: Unit, integration, and functional/acceptance testing. (https://www.quora.com/How-do-you-define-code-quality)

In this context of Code Quality and Unit Testing in JavaScript, Test-driven development (TDD) is the new trend. TDD is a new way to software development which combines test-first development where you write a test before you write just enough production code to fulfill that test and refactoring. What is the primary goal of TDD? One view is the goal of TDD is specification and not validation (Martin, Newkirk, and Kess 2003). In other words, it’s one way to think through your requirements or design before your write your functional code (implying that TDD is both an important agile requirements and agile design technique). Another view is that TDD is a programming technique. As Ron Jeffries likes to say, the goal of TDD is to write clean code that works. I think that there is merit in both arguments, although I lean towards the specification view, but I leave it for you to decide. 

\section{Test-driven development cycle}

TDD can be described with this simple formula: TDD = The steps of test first development (TFD) + a Refactoring. TFD are overviewed in the UML activity diagram of Abbildung 2. With this approach a test should be quickly added. Normally just enough code to fail. After that the tests should be run, often the complete test suite although for sake of speed you may decide to run only a subset, to ensure that the new test does in fact fail. Then, You update your functional code to make it should pass the new tests. The fourth step is to run your tests again. If they fail you need to update your functional code and retest. Once the tests pass the next step is to start over (you may first need to refactor any duplication out of your design as needed, turning TFD into TDD) (http://agiledata.org/essays/tdd.html )

\section{Benefits of TDD}

TDD enables you to take small steps when writing software. This is the most important thing of this concept. This practice is far more productive than attempting to code in large steps. For example, assume some new functional code have been added to code, compiled, and tested. Chances are important that your tests will be broken by defects that exist in the new code. It is much easier to find, and then fix, those defects if you've written two new lines of code than two thousand. The implication is that the faster your compiler and regression test suite, the more attractive it is to proceed in smaller and smaller steps. I generally prefer to add a few new lines of functional code, typically less than ten, before I recompile and rerun my tests.



\section{Karma}

Karma is a test runner for JavaScript that runs on Node.js. It is very well suited to testing any JavaScript projects. Using Karma to run tests using one of many popular JavaScript testing suites (Jasmine, QUnit, Mocha, etc.) and have those tests executed not only in the browsers of your choice, but also on any platform (desktop, phone, tablet.) Karma is highly configurable, integrates with popular continuous integration packages (Jenkins, Travis, and Semaphore) and has excellent plugin support. (http://www.methodsandtools.com/tools/karma.php ) Karma is an Open Source distribution and actually in the version 1.0.

\begin{itemize}
\item Installation - Karma requires Node.js and the Node Package Manager (NPM). So we can install Karma with the simple command:

\textdollar{} npm install -g karma
\item Configuration - A configuration file have first to be created. Then Karma can do what you want. This configuration file can be a JavaScript or a CoffeeScript file. The configuration file can be created manually or generated step by step via the command line:

\textdollar{} karma init

In this way, the created configuration file is then as follows:
\item Application - Karma is started on the console with the following command:

\textdollar{} karma start [path/to/config/file.js]

All tests are now performed and Karma wait for code changes. while tests succeed, Karma will automatically start another test run. In the case of faulty test the result will be show in the console. In this case karma waits for an update or correction of JavaScript code. (https://blog.mayflower.de/4333-Karma-Testrunner-Einfuehrung.html )
\end{itemize}

\section{PhantomJS}

PhantomJS (phantomjs.org) is a headless WebKit scriptable with JavaScript. The latest stable release is version 2.1. PhantomJS is an open source, and is distributed under the BSD license. PhantomJS is created and maintained by Ariya Hidayat, with the help of many contributors.

PhantomJS can be used for many purpurses:

\begin{itemize}
\item Headless web testing: PhantomJS allows the Lightning-fast testing without the browser!
\item Page automation: With the tool it is also possible to Access and manipulate web pages with the standard DOM API, or with any JavaScript library.
\item Screen capture. PhantomJS provides a way to Programmatically capture web contents, including CSS, SVG and Canvas.
\item Network monitoring. PhantomJS facilitates the Automate performance analysis, tracks page loading and exports it as standard HAR format.
\end{itemize}

Among those use cases some features are implemented in PhantomJS:

\begin{itemize}
\item Multiplatform. This webkit is available on major operating systems like Windows, Mac OS X, Linux, and other Unices.
\item It provides a Fast and native implementation of web standards: DOM, CSS, JavaScript, Canvas, and SVG.
\item Pure headless (no X11) on Linux is made available, ideal for continuous integration systems. PhantonJS runs also on Amazon EC2, Heroku, and Iron.io.
\item PhantomJS is easy to install: Download, unpack, and start having fun in just 5 minutes.
\end{itemize}

\subsection{Layout}
\label{ssec:layout}

Format manuscripts two columns to a page, in the manner these
instructions are formatted. The exact dimensions for a page on A4
paper are:

\begin{itemize}
\item Left and right margins: 2.5 cm
\item Top margin: 2.5 cm
\item Bottom margin: 2.5 cm
\item Column width: 7.7 cm
\item Column height: 24.7 cm
\item Gap between columns: 0.6 cm
\end{itemize}

\noindent Papers should not be submitted on any other paper size.
 If you cannot meet the above requirements about the production of your electronic submission, please contact the publication chairs above as soon as possible.


% Removed by KO  we are not accepting printed papers any more!!!
%  Exceptionally,
% authors for whom it is \emph{impossible} to print on A4 paper may use
% \emph{US Letter} paper. In this case, they should keep the \emph{top}
% and \emph{left} margins as given above, use the same column width,
% height and gap, and modify the bottom and right margins as
% necessary. Note that the text will no longer be centered.

\subsection{Fonts}

For reasons of uniformity, Adobe's {\bf Times Roman} font should be
used. In \LaTeX2e{} this is accomplished by putting

\begin{quote}
\begin{verbatim}
\usepackage{times}
\usepackage{latexsym}
\end{verbatim}
\end{quote}
in the preamble. If Times Roman is unavailable, use {\bf Computer
  Modern Roman} (\LaTeX2e{}'s default).  Note that the latter is about
  10\% less dense than Adobe's Times Roman font.


\begin{table}[h]
\begin{center}
\begin{tabular}{|l|rl|}
\hline \bf Type of Text & \bf Font Size & \bf Style \\ \hline
paper title & 15 pt & bold \\
author names & 12 pt & bold \\
author affiliation & 12 pt & \\
the word ``Abstract'' & 12 pt & bold \\
section titles & 12 pt & bold \\
document text & 11 pt  &\\
captions & 11 pt & \\
abstract text & 10 pt & \\
bibliography & 10 pt & \\
footnotes & 9 pt & \\
\hline
\end{tabular}
\end{center}
\caption{\label{font-table} Font guide. }
\end{table}

\subsection{The First Page}
\label{ssec:first}

Center the title, author's name(s) and affiliation(s) across both
columns. Do not use footnotes for affiliations. Do not include the
paper ID number assigned during the submission process. Use the
two-column format only when you begin the abstract.

{\bf Title}: Place the title centered at the top of the first page, in
a 15-point bold font. (For a complete guide to font sizes and styles, see Table~\ref{font-table}) Long titles should be typed on two lines without
a blank line intervening. Approximately, put the title at 2.5 cm from
the top of the page, followed by a blank line, then the author's
names(s), and the affiliation on the following line. Do not use only
initials for given names (middle initials are allowed). Do not format surnames
in all capitals (e.g., use ``Schlangen'' not ``SCHLANGEN'').
Do not format title and section headings in all capitals as well
except for proper names (such as ``BLEU'') that are conventionally
in all capitals.
The affiliation should contain the author's complete address, and if
possible, an electronic mail address. Leave about 2 cm between the
affiliation and the body of the first page.
The title, author names and addresses should be completely
identical to those entered to the electronical paper submission
website in order to maintain the consistency of author information
among all publications of the conference.

{\bf Abstract}: Type the abstract at the beginning of the first
column. The width of the abstract text should be smaller than the
width of the columns for the text in the body of the paper by about
0.6 cm on each side. Center the word {\bf Abstract} in a 12 point bold
font above the body of the abstract. The abstract should be a concise
summary of the general thesis and conclusions of the paper. It should
be no longer than 200 words. The abstract text should be in 10 point font.

{\bf Text}: Begin typing the main body of the text immediately after
the abstract, observing the two-column format as shown in 
the present document. Do not include page numbers.

{\bf Indent} when starting a new paragraph. Use 11 points for text and 
subsection headings, 12 points for section headings and 15 points for
the title. 

\subsection{Sections}

{\bf Headings}: Type and label section and subsection headings in the
style shown on the present document.  Use numbered sections (Arabic
numerals) in order to facilitate cross references. Number subsections
with the section number and the subsection number separated by a dot,
in Arabic numerals. Do not number subsubsections.

{\bf Citations}: Citations within the text appear
in parentheses as~\cite{Gusfield:97} or, if the author's name appears in
the text itself, as Gusfield~\shortcite{Gusfield:97}. 
Append lowercase letters to the year in cases of ambiguity.  
Treat double authors as in~\cite{Aho:72}, but write as in~\cite{Chandra:81} when more than two authors are involved. Collapse multiple citations as in~\cite{Gusfield:97,Aho:72}. Also refrain from using full citations as sentence constituents. We suggest that instead of
\begin{quote}
  ``\cite{Gusfield:97} showed that ...''
\end{quote}
you use
\begin{quote}
``Gusfield \shortcite{Gusfield:97}   showed that ...''
\end{quote}

If you are using the provided \LaTeX{} and Bib\TeX{} style files, you
can use the command \verb|\newcite| to get ``author (year)'' citations.

As reviewing will be double-blind, the submitted version of the papers should not include the
authors' names and affiliations. Furthermore, self-references that
reveal the author's identity, e.g.,
\begin{quote}
``We previously showed \cite{Gusfield:97} ...''  
\end{quote}
should be avoided. Instead, use citations such as 
\begin{quote}
``Gusfield \shortcite{Gusfield:97}
previously showed ... ''
\end{quote}

\textbf{Please do not  use anonymous citations} and  do not include acknowledgements when submitting your papers. Papers that do not conform
to these requirements may be rejected without review. 

\textbf{References}: Gather the full set of references together under
the heading {\bf References}; place the section before any Appendices,
unless they contain references. Arrange the references alphabetically
by first author, rather than by order of occurrence in the text.
Provide as complete a citation as possible, using a consistent format,
such as the one for {\em Computational Linguistics\/} or the one in the 
{\em Publication Manual of the American 
Psychological Association\/}~\cite{APA:83}.  Use of full names for
authors rather than initials is preferred.  A list of abbreviations
for common computer science journals can be found in the ACM 
{\em Computing Reviews\/}~\cite{ACM:83}.

The \LaTeX{} and Bib\TeX{} style files provided roughly fit the
American Psychological Association format, allowing regular citations, 
short citations and multiple citations as described above.

{\bf Appendices}: Appendices, if any, directly follow the text and the
references (but see above).  Letter them in sequence and provide an
informative title: {\bf Appendix A. Title of Appendix}.

\textbf{Acknowledgement} section should go as a last section immediately
before the references.  Do not number the acknowledgement section.

\subsection{Footnotes}

{\bf Footnotes}: Put footnotes at the bottom of the page and use 9
points text. They may be numbered or referred to by asterisks or other
symbols.\footnote{This is how a footnote should appear.} Footnotes
should be separated from the text by a line.\footnote{Note the line
separating the footnotes from the text.}

\subsection{Graphics}

{\bf Illustrations}: Place figures, tables, and photographs in the
paper near where they are first discussed, rather than at the end, if
possible.  Wide illustrations may run across both columns.  Color
illustrations are discouraged, unless you have verified that  
they will be understandable when printed in black ink.

{\bf Captions}: Provide a caption for every illustration; number each one
sequentially in the form:  ``Figure 1. Caption of the Figure.'' ``Table 1.
Caption of the Table.''  Type the captions of the figures and 
tables below the body, using 11 point text.  

\section{Translation of non-English Terms}

It is also advised to supplement non-English characters and terms
with appropriate transliterations and/or translations
since not all readers understand all such characters and terms.
Inline transliteration or translation can be represented in
the order of: original-form transliteration ``translation''.

\section{Length of Submission}
\label{sec:length}

Long papers may consist of up to 8 pages of content, plus two extra pages for references
and short papers may consist of up to 4 pages of content, plus two extra pages for references, in the proceedings.
Papers that do not conform to the specified length and formatting requirements are
subject to be rejected without review.

\section{Other Issues}

Those papers that had software and/or dataset submitted for the review process should also submit it 
with the camera-ready paper. Besides, the software and/or dataset should not be anonymous. 

Please note that the publications of ACL 2013 will be publicly available at ACL Anthology 
(http://aclweb.org/anthology-new/) on July 28th, 2013, one week before the start of the conference. 
Since some of the authors may have plans to file patents related to their papers in the conference, 
we are sending this reminder that July 28th, 2013 may be considered to be the official publication date, 
instead of the opening day of the conference.

\section*{Acknowledgments}

Do not number the acknowledgment section. Do not include this section
when submitting your paper for review.
%\bibliographystyle{acl}
% you bib file should really go here 
%\bibliography{acl2013}

\begin{thebibliography}{}

\bibitem[\protect\citename{Aho and Ullman}1972]{Aho:72}
Alfred~V. Aho and Jeffrey~D. Ullman.
\newblock 1972.
\newblock {\em The Theory of Parsing, Translation and Compiling}, volume~1.
\newblock Prentice-{Hall}, Englewood Cliffs, NJ.

\bibitem[\protect\citename{{American Psychological Association}}1983]{APA:83}
{American Psychological Association}.
\newblock 1983.
\newblock {\em Publications Manual}.
\newblock American Psychological Association, Washington, DC.

\bibitem[\protect\citename{{Association for Computing Machinery}}1983]{ACM:83}
{Association for Computing Machinery}.
\newblock 1983.
\newblock {\em Computing Reviews}, 24(11):503--512.

\bibitem[\protect\citename{Chandra \bgroup et al.\egroup }1981]{Chandra:81}
Ashok~K. Chandra, Dexter~C. Kozen, and Larry~J. Stockmeyer.
\newblock 1981.
\newblock Alternation.
\newblock {\em Journal of the Association for Computing Machinery},
  28(1):114--133.

\bibitem[\protect\citename{Gusfield}1997]{Gusfield:97}
Dan Gusfield.
\newblock 1997.
\newblock {\em Algorithms on Strings, Trees and Sequences}.
\newblock Cambridge University Press, Cambridge, UK.

\end{thebibliography}

\end{document}
\grid
\grid
